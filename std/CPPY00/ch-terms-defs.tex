\section{Terms and Definitions}

For the sake of clarity and consistency, rigorous definitions of terms used in this
Standard are provided herein. For the terms whose definitions are explicitly provided
in this section, they shall not be considered as referring, either implicitly or
explicitly, to any other definitions in any other document. For terms that need to be
defined and whose definitions are not provided in this section, such definitions shall
be given by footnotes in their first occurrence in the text.


\subsection{{\ttfamily compiler}, {\ttfamily interpreter}, and {\ttfamily translator}}

They shall all refer to a program whose main function is to translate source code written
in CpPy into Python code.

\inlinenote Albeit in general context, the three terms refer to different types of
code mechanisms, in the context of this Standard, in which mechanisms for CpPy are
defined, the three terms are used interchangeably.

\inlinenote To reduce ambiguity, the term \texttt{translator} shall be used. However,
when the context is clear, the terms \texttt{compiler} and \texttt{interpreter} may also
be used.


\subsection{{\ttfamily implementation}}

A single or a set of programs that provide the functionalities of a CpPy translator.


\subsection{{\ttfamily implementation-defined behaviour}}

Behaviours that are not specified by this Standard, but are required to be defined by
the implementation. The implementation shall document such behaviours.


\subsection{{\ttfamily shall} and {\ttfamily shall not}}

In this Standard, a statement that uses the term \texttt{shall} indicates a requirement.
A statement that uses the term \texttt{shall not} indicates a prohibition.

A violation of a requirement or a prohibition results in an undefined behaviour.


\subsection{{\ttfamily undefined behaviour}}

Behaviours that are not specified by this Standard.

\inlinenote An \texttt{undefined behaviour} may or may not be a result of an errorneous
program.